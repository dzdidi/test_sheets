\chapter{Abstract}
\label{chap:abstract}
Despite the big amount of domain specific languages and automated code generation tools modern approaches to the test definition and execution still have a high entrance level for people without software development experience. Further, some of the existing approaches provide an opportunity to define tests easily, however they still require developers to create fixtures. 

Test Sheets is an approach for test definitions created by Software Engineering group of the University of Mannheim. It provides  tool-free user experience of test definitions via the use of regular spreadsheets editors.
As a result, it reduces involvement of technical staff into this process to minimum.

This paper describes the process of development and use of system which implements Test Sheets approach in case of figo GmbH. The system design was made with taking in to account next requirements. First, it must be used for asynchronous testing. Second, the system under test represents a real-time software. This two factors leads to the case when one test step can be started before the completion of previous steps while exact time required for test step completion is critical but not known.

The software system implemented within the scope of this research shows the possibility of usage of Test Sheets for asynchronous testing of real-time software.

The research results stated in this paper cover following aspects. System's performance measurements during the code generation and execution. The fit of the system to the indicated problem. User experience analysis after pilot stage of the project. The misfits inherent to the implemented software. The benefits of the system use from different perspectives. 
%\begin{itemize}
%\item 2-3 sentences - current state of art
%\item 1-2 sentences - contribution to improvement
%\item 1-2 sentences - specific result of the paper and main idea behind it
%\item 1 sentences - how result is demonstrated and defend
%\end{itemize}
%
%
%While providing of simple way for test description is a hot topic in software development. There is no software developed for a realization of Test Sheet concept, pragmatic way of defining tests which lays between two extreme paradigms FIT and hard coded test definitions. \\
%
%This paper describes processes of design and implementation of the Test Sheets' concept together with integration of the product to business processes of figo GmbH for a real-time testing/validation of internet banking web pages.\\
%
%Result of this research is following: developed conventions for Test Sheets definitions in particular use case, implemented interpreter from Test Sheets to executable JavaScript  code.
%
%Conventions and the code listing of main module together with example of  executable JavaScript file are provided as well as statistics regarding improvement of user experience and overall system fault prevention improvements.
%
%
%Some feedback from Bianca and Sebastian + statistics regarding user experience improvement

%Implications of this research paper will be helpful for students of computer science disciplines who are interested in combined application of best practices for system design and development from both Object Oriented Programming (OOP) and Functional Programming (FP) approaches. In this paper they will find analysis of suggestions, principles and patterns for design and implementation of scalable and reusable software together with detailed description of the design and implementation process of the Test Sheets paradigm for asyncronous real-time testing.\\ 
%Paper also defines conventions for Test Sheets and indicates test requirements for test implementation and execution.\\
%The design process was made with respect to NodeJS best practices and integral design aspects, design principles and design patterns for OOP together with pipe-and-filter architectural type and application of piping strategies of a FP.\\
%Implementation process described in this paper was performed with following of the Test Driven Development (TDD) rules which guarantees full test coverage together with Clean Code recomendations by Robert C Martin and coding style guide introduced by Airbnb.\\
