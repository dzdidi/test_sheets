\chapter{Introduction}
\label{chap:intro}
This paper shows the possibility of usage if the Test Sheets approach for asynchronous testing of a real-time software. The process of tests definition using Test Sheets is not widely used now and have not been applied for such kind of the problem yet. 
The result of this paper is a proof of concept, complete software product applicable for using Basic Test Sheets not only for asynchronous testing but also for asynchronous testing of real-time software systems.

The wide spread of systems with time constraints imposed on their response time in a web development, caused an appearance of new technologies for asynchronous programming (i.e. node.js, python's tornado or twisted, ruby's event machine etc). This together with increasing level of code reuse raised the speed of software development process, and, as a result, of the velocity of introduction of a new business requirements. However, the processes for definitions of tests did not really change a lot and still have high entry level for people who defines tests and require some development background and thus necessitates direct involvement of software developers in this process. Test Sheets is a new approach developed by Software Engineering group of the University of Mannheim. It combines  tool-free user experience of test definitions for business related companies staff via usage of regular spreadsheets editors and the reduction to minimum of the technical staff involvement in this process.


The work is based on researches of Michael Zhivich and Robert Cunningham in field of software testing. Papers of Robert Glass and Tsai, Fang  and Bi in scope of testing of real-time software systems. Description of asynchronous strategies is based on General Theory of Relativity, the research made by Chris Kowal and researches in field of reactive programming made by Erik Meijer, Conal Eliot and Mark S. Miller. Selection of the strategy for asynchronous event handling in node.js is based on  performance measurements made by Gorgi Kosev. System  architecture and design were made with respect to principles of agile architecture described by Robert  C. Marting and John Dolley. The implementation of the system was made according to Test Driven Development approach and guidances introduced by Robert C. Martin in his book "Clean Code: A Handbook of Agile Software Craftsmanship".


In scope of this paper was designed and implemented guidance for code generation in case of asynchronous testing of real-time software which provides an opportunity to use Test Sheets for testing such the kind of a software systems. The system implemented as a proof of concept for possibility of usage of Basic Test Sheets for asynchronous testing of real-time system is a complete product and is ready to use for testing code implemented in node.js with callback pattern for handling asynchronous events. The conventions (Section \ref{sec:convnetions}) for Test Sheet definitions developed within the scope of this paper put some limitations on number of input/output parameters and reduce their type to javascript object, while in a same time enhance the flexibility by introducing two types of comparison for actual and expected execution result. Within the process of this research and software implementation was made the integration of Basic Test Sheets into the business process of financial technology company.


In this paper were made researches on comparison of Test Sheets to another widely used test definition approaches (Chapter \ref{chap:testing}). Given definition of real-time software with respect to modern state of art in web development and made an investigation regarding specific of a testing process for real-time software systems (Chapter \ref{chap:rt}). The structure of event mechanism of node.js, and strategies for asynchronous event handling together with their performance were investigated (Chapter \ref{chap:async}). Given motivation for system architecture (Chapter \ref{chap:architectureDesign}), design and implementation (Chapter \ref{chap:design}) together with principles and patterns this process was guided by. The performance of the system and its general implication to the business process of the company it was integrated in are given in (Chapter  \ref{chap:performance}). Described limitations with guidance for future work in chapter (Chapter \ref{chap:limits}) The paper is closing with (Chapter \ref{chap:conclusion}).



%\begin{itemize}
%\item What precisely did I answer
%	\subitem what question did I answer
%	\subitem why should the reader care
%	\subitem what larger question does this address
%\item What is my result
%	\subitem What new knowledge have I contributed that reader can use else where
%	\subitem What previous work do I build on
%	\subitem What precisely and in detail my new result
%\item Why should the reader believe in my result
%	\subitem What standard was used to evaluate the claim
%	\subitem What concrete evidence shows that me result satisfies my claim
%\end{itemize}
%
%
%Relevance of the topic and the necessity for scientific investigation: No researches found regarding semi automated tests generation for web page verification.\\
%Practical and theoretical value of the topic: Implementation of enginee for Test Sheets with application of software design and development practices.\\
%Motives for choosing a particular topic:  Necessity of tests defined by non-developers for figo for a real-time testing (will be provided later)\\
%Research problem and why it is worthwhile studying - definition of convention for test sheets definition, usage of test sheets for testing of asynchromous systems in a real-time.\\
%Research objectives - design and development of software for translation of test sheets in to executable java script for testing asyncronous calls to external system.\\
%
%%Research methodology : available research methods, choice of methods, rationale behind the selected selected methods , data sources, research design, data collection instruments and measures to ensure validity and reliability of information, and analytical techniques
%
%Structure of the thesis : A paragraph indicating the main
% Contribution of each chapter and how do they relate
%to the main body of the study
%Limitations of the study\\

%I give a small background why it is important what we are doing.

%State what the problem is.

%Here we write what we are doing in our paper.

%The paper is structured as follows: In the next section (Section \ref{sec:foundations}) we will explain the foundations of bla bla bla. Section \ref{sec:contributions} presents our contribution. [...] The work is closing with related work (Section \ref{sec:related_work}) and conclusions (Section \ref{sec:conclusion}).