\chapter{Introduction}
\label{chap:intro}
This paper shows the case of the Test Sheets' use for asynchronous testing of a real-time software. Tests definition using Test Sheets is not widely used nowadays and has not been applied for such kind of the problems yet.
The end product introduced in this paper is a proof of concept,  applicable for using Basic Test Sheets for asynchronous testing of real-time software systems.

The wide spread of systems with time constraints imposed on their response time in a web development caused an appearance of new technologies for asynchronous programming (i.e. node.js, python's tornado or twisted, ruby's event machine etc). 
This, together with increasing level of code reuse raised the speed of software development process.
As a result, it raised the velocity of a new business requirements introduction.
However, the processes of tests' definitions did not change a lot. 
It still has high entry level for people who define tests requiring them to have software development background.
This fact necessitates direct involvement of software developers in the processes of test definition and editing.
Test Sheets is a new approach developed by Software Engineering group of the University of Mannheim. 
It combines  tool-free user experience of test definitions for business related staff via use of regular spreadsheets editors.
As a result, it reduces involvement of technical staff to minimum.


This work describes the change project: development, adoption, conversion and use processes for an implementation of Test Sheets in figo GmbH. The research is based on works of Michael Zhivich and Robert Cunningham in the field of software testing, as well as papers of Robert Glass and Tsai, Fang  and Bi in scope of real-time software systems testing.
Description of asynchronous strategies analysed in this paper is based on General Theory of Reactivity, the research made by Chris Kowal and researches in field of reactive programming made by Erik Meijer, Cvonal Eliot and Mark S. Miller. 
The selection of the strategy for asynchronous event handling in node.js based on the performance measurements made by Gorgi Kosev. 
System  architecture and design made with respect to the principles of agile architecture described by Robert  C. Marting and John Dolley. 
The implementation of the system made according to the Test Driven Development approach and guidances introduced by Robert C. Martin in his book "Clean Code: A Handbook of Agile Software Craftsmanship".
The user experience collected by polling process based on After Scenario Questionnaire and Post Study System Usability Questionnaire designed by James R Lewis.
The representation of system fit based on Task-Technology Fit model described by Dale Goodhue and Ronald Thompson.

The results of the research made by author within the scope of this paper are stated below.
The guidance of code generation for asynchronous testing of real-time software was created (Section \ref{sec:execOrder}). 
The conventions (Section \ref{sec:convnetions}) for Test Sheet definitions created within the scope of this paper put limitations on a number of input/output parameters and reduce their type to javascript object.
While enhancing the flexibility by introducing two types of comparison of actual and expected execution result. 
Transformation project dedicated to introduction of Basic Test Sheets into the business process of financial technology company described in Section \ref{sec:transP}.
This paper introduces the comparison of Test Sheets with another widely used test definition approaches (Chapter \ref{chap:testing}). 
The definition of real-time software with respect to the modern state of art in web development is given in Chapter \ref{chap:rt} which also describes specifics of a testing process for real-time software systems.
Described architecture (Chapter \ref{chap:architectureDesign}), design and implementation (Chapter \ref{chap:design}) together with principles and patterns this processes were based on. 
The measurements of performance and user experience were made are described in the chapters \ref{chap:perfux} and \ref{chap:UX} representatively. 
The paper describes system fits and misfits together with benefits analysis of system's use (Chapter \ref{chap:fitsBenefits}).
Described limitations with guidance for future work in chapter (Chapter \ref{chap:limits}). The paper is closed with conclusions (Chapter \ref{chap:conclusion}).



%\begin{itemize}
%\item What precisely did I answer
%	\subitem what question did I answer
%	\subitem why should the reader care
%	\subitem what larger question does this address
%\item What is my result
%	\subitem What new knowledge have I contributed that reader can use else where
%	\subitem What previous work do I build on
%	\subitem What precisely and in detail my new result
%\item Why should the reader believe in my result
%	\subitem What standard was used to evaluate the claim
%	\subitem What concrete evidence shows that me result satisfies my claim
%\end{itemize}
%
%
%Relevance of the topic and the necessity for scientific investigation: No researches found regarding semi automated tests generation for web page verification.\\
%Practical and theoretical value of the topic: Implementation of enginee for Test Sheets with application of software design and development practices.\\
%Motives for choosing a particular topic:  Necessity of tests defined by non-developers for figo for a real-time testing (will be provided later)\\
%Research problem and why it is worthwhile studying - definition of convention for test sheets definition, usage of test sheets for testing of asynchromous systems in a real-time.\\
%Research objectives - design and development of software for translation of test sheets in to executable java script for testing asyncronous calls to external system.\\
%
%%Research methodology : available research methods, choice of methods, rationale behind the selected selected methods , data sources, research design, data collection instruments and measures to ensure validity and reliability of information, and analytical techniques
%
%Structure of the thesis : A paragraph indicating the main
% Contribution of each chapter and how do they relate
%to the main body of the study
%Limitations of the study\\

%I give a small background why it is important what we are doing.

%State what the problem is.

%Here we write what we are doing in our paper.

%The paper is structured as follows: In the next section (Section \ref{sec:foundations}) we will explain the foundations of bla bla bla. Section \ref{sec:contributions} presents our contribution. [...] The work is closing with related work (Section \ref{sec:related_work}) and conclusions (Section \ref{sec:conclusion}).