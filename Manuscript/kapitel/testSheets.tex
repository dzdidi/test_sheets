\chapter{Test Sheets}
"Software testing is an important aspect of modern software development.
Testing is performed to ensure that a product or component meets the requirements of all stakeholders.
Just as the requirements themselves tests can vary widely in their nature.
Some tests check whether executing certain code paths results in a correct state or answer while others may check whether a certain code path delivers its results in a specified amount of time.

However writing and evaluating tests is often not much different from programming itself.
Tests are usually written in a formal programming language such as Java/JUnit.
This necessitates that in order to create a test and understand its results knowledge of a formal programming language is required.
This is a significant barrier of entry for stakeholders without a background in IT even though they may be interested in the tests themselves.
Even without deep IT knowledge a stakeholder may still be interested in how well a product performs with regard to her requirements.

Visual test representations such as the UML Testing Profile try to lower the barrier of entry into testing.
However most of the time these visualizations are only partial descriptions of the tests and so do not contain all the desired information for evaluation.

The Software Engineering group has started to develop a new representation for tests called Test Sheets.
The goal is to create a way to define tests which combines the power and completeness of formal programming language with a representation that is easy to understand and work with even for people with little IT knowledge.
This is achieved by representing a test in tabular form as a spreadsheet.
Rows in a Test Sheet represent operations being executed while the columns represent input parameters and expected results.
The actual content of a cell can be made dependent on other cells by addressing them via their location.
This works in a way similar to existing spreadsheet software such as Microsoft Excel.
After executing a test the cells for each expected result is colored according to the result of the test.
A successful test causes cells to become green while failed tests are indicated by red cells."\cite{ts}
 \section{Basic Test Sheets}
"A Test Sheet consists of a name and a class being tested. 
Each row after that represents one method call.
The first column identifies the object being tested while the second column indicates which method is being called on said object. [...]
Input parameters are specified in the columns following the method name up to the invocation line. 
Right after the invocation line the expected return values can be specified."\cite{tsb}

% \paragraph{Non-Linear}
%Similar to the invocation line (double line separating input parameters and expected return values) there is a double line below the last method invocation row.
%Below this line the behavior specification starts. 
%Each row of the behavior specification represents a state of a state machine. 
%The state machine starts in the first state that is being specified right below the double line and stops execution once a state is reached without any valid transitions.

%Each column in a row specifies a transition. 
%A transition consists of a guard, executed invocations and a subsequent state. 
%The starting state has only one transition with no guard, the intermediate state has [...] transitions each with a guard and the terminating state does not have any transitions at all[...].  \cite{tsn}

\section{High Order, Parameterized Test Sheets}
"The actual value used for Parameterized Test Sheets is specified by a Higher-Order Test Sheet as in the example below. 
The Higher-Order Test Sheet references the Parameterized Test Sheet as the 'class' being tested. 
On said pseudo-class it invokes the pseudo-method test followed the by the value to be used as parameter. 
?C in the Parameterized Test Sheet is replaced by the value defined the third column (column C) for each execution.
It is also possible to use more than one parameter. These are defined in the Higher-Order Test Sheet in subsequent columns (D, E, F, etc.) and referenced in the Parameterized Test Sheet via ?D, ?E, ?F, etc"\cite{tsh}
