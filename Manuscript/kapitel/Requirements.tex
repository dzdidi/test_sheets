\chapter{Requirements}
\label{sec:reqiorements}
This section describes requirements introduced to the software by figo GmbH and shows there correlation with the Test Sheets definition provided earlier in this paper.
%and analyses the difference in requirements differences defined by Test Sheets concept and introduced by figo GmbH.

%Firstly the platform of the implementation must be node.js its conceptual differences with such enterprise languages as Java or C# can be understood from the paltform description provided earlier in this paper but it worth to highlight them explicitly. The language of implementation is javascript which is the language with \textit{prototype based inheritance}. The \textit{event-loop mechanism} placed in core of the platform makes it asynchronous. Support of f\textit{unctions as a first-class citizens} provides an ability of confrontational function passing which is a pattern of functional programming. The language is relatively new for implementation of server-side applications.


 
Test Sheets were originally designed for test definitions of OOP languages. And all available examples describes tests definitions for Java Classes. While figo GmbH case requires implementation of tests for nodeJS/casperJS, which are based on JavaScript - Object Oriented, imperative, Functional Oriented programming language with asynchronous information flow.
%Moreover figo GmbH requires input of test results to be recorded in to LogStash logs database. 

Testing of real-time software


The execution requirements from figo GmbH are such that tests defined via Test Sheets should be automatically executed in a time manner (every 2 mins or so), while normal software testing is performed on demand.
The figo's requirement for comparison is such that while defining a Test Sheet user should be able to select from two types of comparison: 1) Strict comparison - complete comparison of objects including both properties' structure and their values. 2) Not-Strict comparison - scheme comparison of objects.

(Probably should go to different section)
Scraping scripts implement callback based approach for handling asynchronous data flow. This provides an opportunity to perform result comparison within the custom callback defined on a implementation stage.
Standard convention for callback definitions limitates number of input parameters of a callback (error, data), while comparison requires compare data parameter with expected outputs from Test Sheet. The opportunity to resolve this issue lays in a JavaScript's support of functions as a class citizens, the function implementation of module for comparison and report: module exports function which invoked with single parameters (expected\_output) / ( + script name) and returns function which is used as a callback for scrapping script, this callback function performs comparison and writes its result to TestSheet or logstash depending on environment in which the program was executed


!!! update conventions and align this list to them!
\begin{enumerate}
	\item javascript
	\item realtime
	\item asynchronous software in combination with reference mechanism of test sheets
	\item high performance for crownjob
	\item two types of comparison
	\item report mechanism
	\item notification mechanism
	
\end{enumerate}