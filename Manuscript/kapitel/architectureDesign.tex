\chapter{Architecture} 
\label{sec:architectureDesign}
The system consist of three piped object streams \ref{fig:tsGenArch}. Streams piping implements the idea of pipe '|' in Unix systems invented by Douglas Mcllroy. It enables the output of one program to be connected to the input of the next program. \textit{Object stream (Stream in a object mode)} is a stream in which data treated as a sequence of discrete javascript objects. Further, piping of object streams allows to perform parallel executions which can be beneficial from the performance perspective. 

\begin{figure}[ht]
	\label{fig:tsGenArch}
	\centering
	\includegraphics[scale=0.75]{grafiken/TSGeneratorArchitecture}
	\caption{Information flow}
\end{figure}

\begin{enumerate}
	\item \textbf{\textit{Read Stream}} accepts directory address and \textit{pulls} content of all .xlsx files together with their metadata  from this directory including nested directories;
	\item \textbf{\textit{Transform Stream}} accept data object with file name, content, metadata from upstream, creates TestSheet schema and generates content of .js file implementing \textit{interpreter pattern}; 
	\item \textbf{\textit{Write Stream}} \textit{pulls} data from upstream perform attempt to read representative .js file from specified folder if file exists and its last update date is older then for .xlsx file the next step will be skipped;
	\item \textbf{\textit{Write Stream}} if file does not exist or its last update date is earlier then last update date of .xlsx file it creates/overwrites .js file.
\end{enumerate}

Pipe method of streams provide developers with opportunity to chain streams implementing different piping patterns:
\begin{enumerate}
	\item \textit{Combining} - encapsulation of sequentualy connected streams in to single looking stream with single I/O points and single error handling mechanism by pipeing readable stream in to writable stream;
	\item \textit{Forking/Merging} - piping single readable in to multiple writable streams /  piping multiple readable streams in to single writable stream;
	\item \textit{Multiplexing/Demultiplexing} - forking and merging pattern which provides shared communication channel for entities from different streams, analogy can be computer networks.
\end{enumerate}

As was already described streams are deferred analog of arrays, which allows to perform such operations as mapping, reducing and filtering. The process of transformation of xlsx files to js files in this application is treated as mapping process in general case, and filtering for avoiding of redundant file's overwriting.

The folder/file structure of the application looks as following:
\begin{itemize}
	\item index.js
	\item package.json
	\item ReadMe.md
	\item lib/
	\begin{itemize}
		\item scheme/
		\begin{itemize}
			\item index.js
			\item execution\_scheme.js
			\item order.js
			\item scheme.js
		\end{itemize}
		\item stream/
		\begin{itemize}
			\item index.js
			\item read\_stream.js
			\item transform\_stream.js
			\item write\_stream.js
		\end{itemize}
		\item template/
		\begin{itemize}
			\item index.js
		\end{itemize}
	\end{itemize}
	\item test/
	\begin{itemize}
		\item read\_stream.js
		\item write\_stream.js
		\item transform\_stream.js
		\item scheme.js
		\item execution\_scheme.js
		\item order.js
		\item template.js
		\item doublers/
		\begin{itemize}
			\item TestSheetOjbect.js
			\item TestSheet.xlsx
			\item TestSheet.js
		\end{itemize}
	\end{itemize}
	\item node\_modules/
\end{itemize}
Creation of folders for scheme and template directories is made for purpose of expansion in case of adding Non-linear and/or HigherOrder Test Sheets. 
The entry point of the system ./index.js looks as following (Listing: \ref{index}):

\lstinputlisting[
language=Javascript, numbers=left, stepnumber=5, firstnumber=1, breaklines=true, 
basicstyle=\footnotesize,
numberstyle=\tiny,
caption={index.js},
captionpos=b,
label=index
]
{code/index.js.txt}

